\documentclass{article}

\usepackage[utf8]{inputenc}
\usepackage[french]{babel}
\usepackage{color}
\usepackage{listings}
\usepackage{graphicx}
\usepackage{bold-extra} % permet le \bf\ttfamily de keywordstyle %
                        % INCOMPATIBLE AVEC ae et aeguill, permet le
                        % \tt\bf
\usepackage{alltt}
\lstset{language=[Objective]Caml}
\lstset{%
  escapeinside={(*@}{@*)},%
  morecomment=*[n][\it\ttfamily]{(*}{*)},% reconnaît mots-clés, commentaires...
  moredelim=[is][\it]{/*}{*/},% éviter keyword style in comment
  flexiblecolumns=true,%
  mathescape=true,%
  basicstyle=\tt\small,%
  keywordstyle=\bf\ttfamily,%
  commentstyle=\it\ttfamily,%
%  backgroundcolor=\color{grey2},%
}

\usepackage{geometry}
\geometry{legalpaper, landscape, margin=1cm}

\usepackage{amsthm}% pour le theoremstyle
% Pour que le newtheorem exercise de exercise.sty soit en normal et
% pas en italic:
\newtheoremstyle{exo}% name
  {10pt}%Space above        
  {10pt}% Space below
  {}% Body font
  {}% Indent amount (empty = no indent, \parindent = normal paragraph indent)
  {\bf\large}% Theorem head font
  {:}% Punctuation after theorem head
  {\newline}% Space after theorem head ({ } = normal interword space; \newline = linebreak)
  {}% Theorem head spec (can be left empty, meaning 'normal')
\theoremstyle{exo}
\usepackage[noanswer]{exercise} %now/hide
\theoremstyle{plain}%retour au style normal pour les théorème.
\usepackage{amssymb}
\newcommand{\setN}{{\mathord{\mathbb N}}}

\usepackage{verbatim}
\let\answer=\comment
\let\endanswer=\endcomment 
%\renewenvironment{answer}{\comment}{\endcomment}
\renewcommand{\ExerciseName}{Exercice}
\usepackage{times}


\usepackage{tikz,pgf}
\usetikzlibrary{shapes,arrows,decorations.markings,calc,positioning,decorations.pathreplacing}
\pgfsetxvec{\pgfpoint{1mm}{0cm}}
\pgfsetyvec{\pgfpoint{0cm}{1mm}}
\usetikzlibrary{backgrounds}
\pgfdeclarelayer{myback}
\pgfsetlayers{background,myback,main}

\newlength{\cellw}
\newlength{\cellh}
\setlength{\cellh}{1mm}
\setlength{\cellw}{3.5mm}
\usepackage{etoolbox}% http://ctan.org/pkg/etoolbox
\tikzstyle{barreB}=[rectangle,draw=white,text width=#1\cellw,minimum height=\cellh,fill=white,inner sep=0,outer sep=0] %,inner sep=2.5mm
\tikzstyle{barreContour}=[rectangle,draw=black,text width=#1\cellw,minimum height=\cellh,inner sep=0,outer sep=0] %,inner sep=2.5mm
\tikzstyle{barreContourgris}=[rectangle,draw=gray,text width=#1\cellw,minimum height=\cellh,inner sep=0,outer sep=0] %,inner sep=2.5mm
\tikzstyle{barreGrisee}=[rectangle,draw=black,fill=gray,text width=#1\cellw,minimum height=\cellh,inner sep=0,outer sep=0] %,inner sep=2.5mm



\definecolor{fillbarre}{gray}{0}
%\definecolor{fillbarre}{gray}{1}

\newcounter{tabcnt}
\newcommand{\barre}[2][]{
  \ifthenelse{#1=1}{\node[barreContour=1] at (0,#2\cellh){~};}
  {\ifthenelse{#1=0}{\node[barreB=1] at (0,#2\cellh){~};}
    {\ifthenelse{#1=2}{\node[barreGrisee=1] at (0,#2\cellh){~};}
      {\node[barreContourgris=1] at (0,#2\cellh){~};}
      }}
}

\newcommand{\tab}[1]{%
  \setcounter{tabcnt}{0}
  %\def\nextitem{\def\nextitem{}}% Separator
  \renewcommand*{\do}[1]{\barre[##1]{\value{tabcnt}}\stepcounter{tabcnt}}% How to process each item
  \docsvlist{#1}% Process list
  \draw[-](-.5\cellw,\value{tabcnt}\cellh-.5\cellh) to (.5\cellw,\value{tabcnt}\cellh-.5\cellh){};
  \draw[-](-.5\cellw,0) to (-.5\cellw,\value{tabcnt}\cellh-.5\cellh){};
  \draw[-](.5\cellw,0) to (.5\cellw,\value{tabcnt}\cellh-.5\cellh){};

  \node[draw=black,rectangle,text width=3\cellw,minimum height=12\cellh](cmframe) at (-\cellw,5\cellh) {};
}


\newlength{\sparkcellh}
\setlength{\sparkcellh}{1.35\cellh}
\newlength{\pilew}
\newlength{\pileh}
\setlength{\pileh}{15\cellh}
\setlength{\pilew}{3\cellw}

\newcommand{\tabspark}[1]{%
  \setcounter{tabcnt}{0}
  %\def\nextitem{\def\nextitem{}}% Separator
  \renewcommand*{\do}[1]{\barre[##1]{\value{tabcnt}}\stepcounter{tabcnt}}% How to process each item
  \docsvlist{#1}% Process list
  \draw[-](-.5\cellw,\value{tabcnt}\cellh-.5\cellh) to (.5\cellw,\value{tabcnt}\cellh-.5\cellh){};
  \draw[-](-.5\cellw,0) to (-.5\cellw,\value{tabcnt}\cellh-.5\cellh){};
  \draw[-](.5\cellw,0) to (.5\cellw,\value{tabcnt}\cellh-.5\cellh){};

  \node[rectangle,thick,draw=black,text width=\pilew,minimum height=\pileh,inner sep=0mm] (sprkframe) at (0,.4\pileh) {};
}






\newcommand{\baseCminor}{
  \tab{2,1,1,1,0,0,0,0,0};
  \node (stkptr) at (-3.5\cellw,12\cellh) {{\scriptsize\texttt{stkptr}}};
  \node (hautgauche) at (-3\cellw,12\cellh) {};
  \node (basgauche) at (-3\cellw,-45\cellh) {};
  \node (cmcentre) at (-1\cellw,0mm) {};
  \node (branchlvl4) at (0,0) {};
  \begin{scope}[xshift=\cellw,yshift=-15\cellh]
    \tab{2,1,1,1,0,0,0,0,0};
    \node (branchlvl3) at (0,0) {};
    \begin{scope}[xshift=\cellw,yshift=-15\cellh]
      \tab{2,1,1,1,0,0,0,0,0};
      \node (branchlvl2) at (0,0) {};
      \begin{scope}[xshift=\cellw,yshift=-15\cellh]
        \tab{2,1,1,1,0,0,0,0,0};
        \node (branchlvl1) at (0,0) {};
      \end{scope}
    \end{scope}
  \end{scope}
  \draw[->,gray,outer sep=0mm] (stkptr) to[out=-90,in=180] (branchlvl4);
  \draw[->,gray,outer sep=0mm] (branchlvl4) to[out=-120,in=180] (branchlvl3);
  \draw[->,gray,outer sep=0mm] (branchlvl3) to[out=-120,in=180] (branchlvl2);
  \draw[->,gray,outer sep=0mm] (branchlvl2) to[out=-120,in=180] (branchlvl1);
}

\newcommand{\pileSpark}[1]{
  \setlength{\pileh}{15\cellh}
  \setlength{\pileh}{#1\pileh} %% Apply arg
  \setlength{\pilew}{3\cellw}
  \begin{pgfonlayer}{myback}
    \node[rectangle,thick,draw=black,fill=white,text
    width=\pilew,minimum height=\pileh,inner sep=0mm] at (0,.5\pileh-1pt) {};
  \end{pgfonlayer}

  %\draw[-](-.5\pilew,\pileh) to (.5\pilew,\pileh){};
  %\draw[-](-.5\pilew,0) to (-.5\pilew,\pileh){};
  %\draw[-](.5\pilew,0) to (.5\pilew,\pileh){};
  %\draw[-](-.5\pilew,0) to (.5\pilew,0){};
}

\pagestyle{empty}
\begin{document}

\begin{tikzpicture}[thick,shorten >=0mm,shorten <=0mm,draw=black]
  %%%%%%%%%%%%%%%%%%%%%%%%%%%%%%%%%%%%%%%%%%%%%%%%%%%%%%%%%%%%%%%%%%%%
  %%%%%% step 1 Cminor
  \baseCminor
  %%%%%%%%%%%%%%%%%%%%%%%%%%%%%%%%%% 
  % step 1 spark
  \begin{scope}[yshift=-70\cellh]
    \tabspark{1,1,1,0,0,0,0,0,0}; \node (branchlvl4) at (0,0) {};
    \node[yshift=15\cellh](hautgauche1) at (hautgauche) {};
    \draw [decorate,decoration={brace,amplitude=10pt},xshift=-4pt,yshift=0pt]
    (-\cellw,-16\cellh) -- (-\cellw,13\cellh) node [black,midway,xshift=-1cm]
    {\scriptsize \verb!intact_s!};
    \begin{scope}[yshift=-15\cellh]
      \tabspark{1,1,1,0,0,0,0,0,0}; \node (branchlvl3) at (0,0) {};
      \begin{scope}[yshift=-15\cellh]
        \tabspark{1,1,1,0,0,0,0,0,0}; \node (branchlvl2) at (0,0) {};
        \draw [decorate,decoration={brace,amplitude=10pt},xshift=-4pt,yshift=0pt]
        (-\cellw,-16\cellh) -- (-\cellw,13\cellh) node [black,midway,xshift=-1cm] 
        {\scriptsize \verb!suffix_s!};
        \begin{scope}[yshift=-15\cellh]
          \tabspark{1,1,1,0,0,0,0,0,0}; \node (branchlvl1) at (0,0) {};
          \node (sprkbas) at (sprkframe.south) {};
          \begin{scope}[yshift=-1\cellh]
            \pileSpark{4}
          \end{scope}
        \end{scope}
      \end{scope}
    \end{scope}
  \end{scope}
  \draw[-,dashed](cmframe.east|-hautgauche1) to (cmframe.east|-sprkbas);

  %%%%%%%%%%%%%%%%%%%%%%%%%%%%%%%%%%%%%%%%%%%%%%%%%%%%%%%%%%%%%%%%%%%%%%%
  %%%%%% step 2 Cminor
  \begin{scope}[xshift=4\cellw]
    \begin{scope}[xshift=6\cellw,yshift=-15\cellh]
      \tab{1,1,1,1,0,0,0,0,0};
      \node[yshift=15\cellh](hautgauche1) at (hautgauche) {};
      \node (branchnewlvl3) at (0,0) {};
      \node (newcentre) at (-1.5\cellw,12\cellh) {};
      \node (spbproc) at (-3.5\cellw,14\cellh) {\scriptsize\texttt{spb\_proc}};
      \draw[->,gray,outer sep=0mm] (spbproc) to[out=-90,in=180] (branchnewlvl3);
      \draw[->,ultra thick,gray,outer sep=0mm] (hautgauche-|branchlvl4.west)
      to[out=20,in=160] node[above](mid1){\color{black}\scriptsize \texttt{malloc~+set\_locals}} (newcentre|-hautgauche);
      \node (mlocenv_init) at (newcentre|-hautgauche1){\texttt{m,locenv\_init}};
    \end{scope}
    %%%%%%%%%%%%%%%%%%%%%%%%%%%%%%%%%% 
    %%%%%% step 2 spark
    %% No spark state corresponds to step 2


    %%%%%%%%%%%%%%%%%%%%%%%%%%%%%%%%%%%%%%%%%%%%%%%%%%%%%%%%%%%%%%%%%%%%%%% 
    %%%%%% step 3 Cminor
    \node (m) at (cmcentre|-hautgauche1){\texttt{m,locenv}};
    \begin{scope}[xshift=13\cellw,thick]
      \baseCminor
      \begin{scope}[xshift=7\cellw,yshift=-15\cellh]
        \tab{2,1,1,1,0,0,0,0,0};
        \node (branchnewlvl3postchain) at (0,0) {};
        \node (haut2) at (hautgauche-|branchlvl4) {};
        \draw[->,ultra thick,gray,outer sep=0mm] (newcentre|-hautgauche)
        to[out=20,in=160] node[above](mid2){\color{black}\scriptsize \texttt{(Sstore \dots chaining\_aram)}}(haut2) ;
        \node (newcentre2) at (-12\cellw,13\cellh) {};
        \node (mprocpreinit) at (branchlvl2|-hautgauche1) {
          \begin{minipage}{5\cellw}
            \texttt{m\_proc\_pre\_init, locenv\_init}
          \end{minipage}};
        \node (spbproc) at (-3.5\cellw,15\cellh) {\scriptsize\texttt{spb\_proc}};
        \draw[->,gray,outer sep=0mm] (spbproc) to[out=-90,in=180] (branchnewlvl3postchain);
        \draw[-,dashed](mid2|-hautgauche1) to (mid2|-basgauche);
      \end{scope}
      \draw[->,gray,outer sep=0mm] (branchnewlvl3postchain) to[out=-120,in=0] (branchlvl2);
      %%%%%%%%%%%%%%%%%%%%%%%%%%%%%%%%%% 
      %%%%%% step 3 spark
      %% No spark state corresponds to step 2
      \begin{scope}[yshift=-85\cellh]
        \tabspark{1,1,1,0,0,0,0,0,0};
        \node (newframe) at (0,0) {};
        \draw [decorate,decoration={brace,amplitude=10pt},xshift=-4pt,yshift=0pt]
        (sprkframe.south west) -- (sprkframe.north west) node [black,midway,xshift=-.5cm] 
            {\scriptsize \verb!f!};
        \node (xxx) [above=1.5cm]% right is relative to previous node
        {\scriptsize copyIn};
        \begin{scope}[xshift=6\cellw]
          \tabspark{1,1,1,0,0,0,0,0,0};
          \node (newframe) at (0,0) {};
          \draw [decorate,decoration={brace,amplitude=10pt},xshift=-4pt,yshift=0pt]
          (sprkframe.south west) -- (sprkframe.north west) node [black,midway,xshift=-.5cm] 
          {\scriptsize \verb!f1!};
          \node (xxx) [above=1.5cm]% right is relative to previous node
          {\scriptsize evalDecl};
        \end{scope}
      \end{scope}


      %%%%%%%%%%%%%%%%%%%%%%%%%%%%%%%%%%%%%%%%%%%%%%%%%%%%%%%%%%%%%%%%%%%%%%% 
      %%%%%% step 4 Cminor
      \draw[-,dashed,thick](cmframe.east|-hautgauche1) to (cmframe.east|-basgauche);
      \begin{scope}[xshift=15\cellw,thick]
        \baseCminor
        \begin{scope}[xshift=7\cellw,yshift=-15\cellh]
          \tab{2,1,1,1,0,0,0,0,0};
          \node (haut3) at (hautgauche-|branchlvl4) {};
          \node (branchnewlvl3prebdy) at (0,0) {};
          \draw[->,gray,outer sep=0mm] (branchnewlvl3prebdy) to[out=-120,in=0] (branchlvl2);
          \begin{pgfonlayer}{background}[thick]
            \draw[->,ultra thick,gray,outer sep=0mm] (haut2)
            to[out=10,in=170] node[above](mid3){\color{black}\scriptsize \texttt{(Sseq s\_parms s\_locvarinit)}} (hautgauche);
          \end{pgfonlayer}
          \node (spbproc) at (-3.5\cellw,15\cellh) {\scriptsize\texttt{spb\_proc}};
          \draw[->,gray,outer sep=0mm] (spbproc) to[out=-90,in=180] (branchnewlvl3prebdy);
          \draw[-,dashed](cmframe.east|-hautgauche1) to (cmframe.east|-sprkbas);
        \end{scope}          
        %%%%%%%%%%%%%%%%%%%%%%%%%%%%%%%%%% 
        %%%%%% step 4 spark
        \begin{scope}[yshift=-70\cellh]
          \begin{pgfonlayer}{background}[thick,shorten >=0mm,shorten <=0mm]
            \begin{scope}[xshift=-1.7\cellw,yshift=-2mm,thick,draw=gray]
              \tabspark{3,3,3,0,0,0,0,0,0};
              \node (branchlvl4) at (0,0) {};
              \begin{scope}[yshift=-15\cellh]
                \tabspark{3,3,3,0,0,0,0,0,0};
                \node (branchlvl3prebdy) at (0,0) {};
              \end{scope}
            \end{scope}
          \end{pgfonlayer}
          \begin{scope}[yshift=-15\cellh,thick]
            \tabspark{1,1,1,0,0,0,0,0,0};
            \node (branchlvl4prebdy) at (sprkframe.north) {};
            \draw [decorate,decoration={brace,amplitude=10pt},xshift=-4pt,yshift=0pt]
            (\pilew,\pileh-\cellh) -- (\pilew,-\cellh) node [black,midway,xshift=.5cm] 
            {\scriptsize \verb!f1!};
            \begin{scope}[yshift=-15\cellh]
              \tabspark{1,1,1,0,0,0,0,0,0};
              \node (branchlvl2) at (0,0) {};
              \begin{scope}[yshift=-15\cellh]
                \tabspark{1,1,1,0,0,0,0,0,0};
                \node (branchlvl1) at (0,0) {};
                \begin{scope}[yshift=-1\cellh]
                  \pileSpark{3}
                \end{scope}
              \end{scope}
            \end{scope}
          \end{scope}
        \end{scope}


        %%%%%%%%%%%%%%%%%%%%%%%%%%%%%%%%%%%%%%%%%%%%%%%%%%%%%%%%%%%%%%%%%%%%%%% 
        %%%%%% step 5
        \begin{scope}[xshift=15\cellw,thick]
          \baseCminor
          \begin{scope}[xshift=7\cellw,yshift=-15\cellh]
            \tab{2,1,1,1,0,0,0,0,0};
            \node (haut4) at (hautgauche-|branchlvl4) {};
            \node (branchnewlvl3postbdy) at (0,0) {};
            \draw[->,gray,outer sep=0mm] (branchnewlvl3postbdy) to[out=-120,in=0] (branchlvl2);
            \node (spbproc) at (-3.5\cellw,15\cellh) {\scriptsize\texttt{spb\_proc}};
            \draw[->,gray,outer sep=0mm] (spbproc) to[out=-90,in=180] (branchnewlvl3postbdy);
          \end{scope}
          \begin{pgfonlayer}{background}[thick,shorten >=0mm,shorten <=0mm]
            \draw[->,ultra thick,gray,outer sep=0mm] (haut3)
            to[out=10,in=170] node[above](mid4){\color{black}\scriptsize \texttt{s\_pbdy}} (hautgauche);
          \end{pgfonlayer}
          % \draw[->,ultra thick,gray,outer sep=0mm,shorten >=4\cellw,shorten <=4\cellw] (branchnewlvl3prebdy)
          % to[out=20,in=160] node[above]{\color{black}\scriptsize exec\_stmt s\_pbdy} (branchlvl3);
          \draw[-,dashed](cmframe.east|-hautgauche1) to (cmframe.east|-sprkbas);
          \node[xshift=1ex] (mpostbdy) at (branchlvl4|-hautgauche1) {\texttt{m\_postbdy, locenv\_postbdy}};

          %%%%%%%%%%%%%%%%%%%%%%%%%%%%%%%%%% 
          %%%%% step 5 spark
          \begin{scope}[yshift=-70\cellh]
            \begin{pgfonlayer}{background}[thick,shorten >=0mm,shorten <=0mm]
              \begin{scope}[xshift=-1.7\cellw,yshift=-2mm,thick,draw=gray]
                \tabspark{3,3,3,0,0,0,0,0,0};
                \node (branchlvl4) at (0,0) {};
                \begin{scope}[yshift=-15\cellh]
                  \tabspark{3,3,3,0,0,0,0,0,0};
                  \node (branchlvl3postbdy) at (0,0) {};
                \end{scope}
              \end{scope}
            \end{pgfonlayer}
            \begin{scope}[yshift=-15\cellh,thick]
              \tabspark{1,1,1,0,0,0,0,0,0};
              \node (branchlvl4postbdy) at (sprkframe.north) {};
              \draw[decorate,decoration={brace,amplitude=10pt},xshift=-4pt,yshift=0pt]
              (\pilew,\pileh-\cellh) -- (\pilew,-\cellh) node [black,midway,xshift=1.2cm] 
              {
                \begin{minipage}{1.5cm}
                  \scriptsize \verb!f1'_l! \verb!++ f1'_p!
                \end{minipage}
              };
              \draw[->,ultra thick,gray,outer sep=0mm,shorten >=3\cellw,shorten <=3\cellw] (branchlvl4prebdy.north)
              to[out=10,in=170] node[above]{\color{black}\scriptsize \texttt{procedure\_statements}} (branchlvl4postbdy);
              \begin{scope}[yshift=-15\cellh]
                \tabspark{1,1,1,0,0,0,0,0,0};
                \node (branchlvl2) at (0,0) {};
                \draw [decorate,decoration={brace,amplitude=10pt},xshift=-4pt,yshift=0pt]
                (-\cellw,-16\cellh) -- (-\cellw,13\cellh) node [black,midway,xshift=-1.2cm] 
                {\scriptsize \verb!suffix_s'!};
                \begin{scope}[yshift=-15\cellh]
                  \tabspark{1,1,1,0,0,0,0,0,0};
                  \node (branchlvl1) at (0,0) {};
                  \begin{scope}[yshift=-1\cellh]
                    \pileSpark{3}
                  \end{scope}
                \end{scope}
              \end{scope}
            \end{scope}
          \end{scope} % step 5 spark

          %%%%%%%%%%%%%%%%%%%%%%%%%%%%%%%%%%%%%%%%%%%%%%%%%%%%%%%%%%%%%%%%%%%%%%% 
          %%%%%% step 6 Cminor
          \begin{scope}[xshift=15\cellw,thick]
            \baseCminor
            \begin{scope}[xshift=7\cellw,yshift=-15\cellh]
              \tab{2,1,1,1,0,0,0,0,0};
              \node (haut5) at (hautgauche-|branchlvl4) {};
              \node (branchnewlvl3postbdy) at (0,0) {};
              \draw[->,gray,outer sep=0mm] (branchnewlvl3postbdy) to[out=-120,in=0] (branchlvl2);
              \node (spbproc) at (-3.5\cellw,15\cellh) {\scriptsize\texttt{spb\_proc}};
              \draw[->,gray,outer sep=0mm] (spbproc) to[out=-90,in=180] (branchnewlvl3postbdy);
            \end{scope}
            \begin{pgfonlayer}{background}[thick,shorten >=0mm,shorten <=0mm]
              \draw[->,ultra thick,gray,outer sep=0mm] (haut4)
              to[out=10,in=170] node[above](mid5){\color{black}\scriptsize \texttt{s\_copyout}} (hautgauche);
            \end{pgfonlayer}
            \node (mpostcpout) at (branchlvl2|-hautgauche1) {\texttt{m\_postcpout, locenv\_postcpout}};

            %%%%%%%%%%%%%%%%%%%%%%%%%%%%%%%%%% 
            %%%%%% step 6 spark
            \begin{scope}[yshift=-70\cellh]
              \tabspark{1,1,1,0,0,0,0,0,0};
              \node (branchlvl4) at (0,0) {};
              \node (branchlvl4postcpout) at (sprkframe.west|-branchlvl4) {};
              \node[yshift=15\cellh](hautgauche1) at (hautgauche) {};
              %\draw[decorate,decoration={brace,amplitude=10pt},xshift=-4pt,yshift=0pt]
              %(-\cellw,-16\cellh) -- (-\cellw,13\cellh) node [black,midway,xshift=-1cm]
              %{\scriptsize \verb!intact_s!};
              \begin{scope}[yshift=-15\cellh]
                \tabspark{1,1,1,0,0,0,0,0,0}; \node (branchlvl3) at (0,0) {};
                \begin{scope}[yshift=-15\cellh]
                  \tabspark{1,1,1,0,0,0,0,0,0}; \node (branchlvl2) at (0,0) {};
                  \draw [decorate,decoration={brace,amplitude=10pt},xshift=-4pt,yshift=0pt]
                  (-\cellw,-16\cellh) -- (-\cellw,44\cellh) node [black,midway,xshift=-1cm] 
                  {\scriptsize \verb!s'!};
                  \begin{scope}[yshift=-15\cellh]
                    \tabspark{1,1,1,0,0,0,0,0,0}; \node (branchlvl1) at (0,0) {};
                    \node (sprkbas) at (sprkframe.south) {};
                    \begin{scope}[yshift=-1\cellh]
                      \pileSpark{4}
                    \end{scope}
                  \end{scope}
                \end{scope}
              \end{scope}
            \end{scope} % step 6 spark
            \draw[->,ultra thick,gray,outer sep=0mm,shorten >=\cellw,shorten <=\cellw] (branchlvl4postbdy.north)
            to[out=10,in=170] node[above]
            {\begin{minipage}{10\cellw}
                \color{black}\scriptsize \texttt{copy\_out (intact\_s
                  ++ suffix\_s') f1'\_p proc\_params\_prof args}
              \end{minipage}}
            (branchlvl4postcpout);
          \end{scope}  % step 6 cminor
        \end{scope} % step 5 cminor
      \end{scope} % step 4 cminor
    \end{scope} % step 3 cminor
  \end{scope}% step 2 cminor

\end{tikzpicture}


\end{document}

%%% Local Variables:
%%% mode: latex
%%% TeX-master: t
%%% End:
